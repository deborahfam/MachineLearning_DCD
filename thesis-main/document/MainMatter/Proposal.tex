\chapter{Propuesta}\label{chapter:proposal}

\section{Metodología}\label{sec:method}
%-----------------------------------------------------------------------------------
La metodología a utilizar en este trabajo es la siguiente: 
\begin{enumerate}
    \item Selección y representación del problema y sus datos de forma.
    \item Datos de forma compatible con los algoritmos de aprendizaje automático (importación de imágenes, hojas de verdad en formato csv, etc., registros de verdad en formato csv).
    \item Preprocesamiento de los datos y las imágenes (eliminar el pelo de las fotos, eliminar las luces y las sombras, dividir los canales de las fotos y aplicar cierto grado de desenfoque para reducir el pelo).
    \item Modificación y tratamiento de estos datos (selección de los tres conjuntos para el aprendizaje supervisado, normalización de estas entradas, normalización de la ponderación de cada una de las clasificaciones).
    \item Entrenamiento y comparación de los modelos de aprendizaje (utilización de SVM y del modelo EfficientNetB5 para entrenar el modelo, utilización de ERS para ajustar la forma del aprendizaje y detectar la forma del aprendizaje) para ajustar la forma del aprendizaje y detectar el criterio de parada del proceso de entrenamiento. 
    \item Elaboración de gráficos para proporcionar un criterio visual de los resultados obtenidos (utilizar el modelo obtenido frente al conjunto de pruebas y mostrar gráficos estadísticos de dispersión y errores en los resultados del modelo, extraer una estimación de los resultados del modelo, extraer una estimación de la eficacia). 
\end{enumerate}