\chapter*{Introducción}\label{chapter:introduction}
\addcontentsline{toc}{chapter}{Introducción}

La medicina es una rama de la ciencia en constante evolución. Cada vez se desarrollan más métodos innovadores para mejorar el diagnóstico precoz 
de enfermedades potencialmente mortales cuya detección temprana puede prolongar considerablemente la esperanza de vida de los pacientes. En la actualidad, 
la detección del cáncer de piel es un proceso estrictamente humano que depende del oncólogo y de una prueba que en algunos países puede resultar cara 
(biopsia), aunque en algunos países ya se está intentando utilizar modelos computacionales para la detección.

La inteligencia artificial (IA) y el aprendizaje automático (ML) han experimentado un progreso significativo en los últimos años, y su aplicación 
en la detección de enfermedades, como el cáncer de piel, ha generado un gran interés en la comunidad científica. El ML y la IA han demostrado ser 
herramientas valiosas en la biología y la medicina, ayudando a los científicos a determinar fármacos potencialmente mejores y a entender de manera 
más profunda el comportamiento de la vida celular.

El avance en la detección de cáncer mediante el uso de la inteligencia artificial (IA) y el aprendizaje automático (ML) ha sido significativo en 
las últimas décadas. A continuación, se presenta una cronología de cómo ha ido avanzando la detección de cáncer con ML:

\begin{enumerate}
    \item 2016: Se publicaron estudios sobre el uso de modelos de ML para la predicción del sobrevivencia del cáncer de mama. \cite{ml-breast-cancer}
    \item 2018: Se realizaron investigaciones sobre el uso de técnicas de ML en el análisis de imágenes médicas para el diagnóstico y tratamiento de 
    diferentes tipos de cáncer, incluido el cáncer de piel. \cite{ml-cancer-pred-diag}
    \item 2022: Se publicaron revisiones sobre la aplicación de enfoques de ML en la segmentación y clasificación de lesiones de piel en imágenes 
    dermoscopicas. \cite{ml-techniques-review}
    \item 2023: Se publicaron estudios sobre el uso de la IA y el ML en la predicción de la respuesta a la inmunoterapia en el cáncer de pulmón.\cite{ai-lung-cancer}
\end{enumerate}

Al utilizar ML en la detección de cáncer de piel, se puede acelerar el proceso de diagnóstico, lo que permite a los médicos tomar decisiones de 
tratamiento más rápidas y efectivas nature.com. Esto es especialmente importante en el cáncer de piel, ya que la detección temprana y el tratamiento 
temprano pueden mejorar significativamente las perspectivas de supervivencia y calidad de vida de los pacientes.

La dermatología y la oncologic pueden acercarse a la informática a través de la detección precoz de afecciones cancerosas de la piel, con una foto de
una lesión en una fase temprana y un conjunto de datos de imágenes de lesiones diagnosticadas, 
más tarde, el ordenador puede ser entrenado para alertar a los médicos de posibles afecciones antes de que se desarrollen y se conviertan en letales.

Según la Sociedad Americana del Cáncer, Mitchell et al. (2020) \cite{mitchell}, los cánceres de piel melanoma tienen un estadio temprano 
formalmente llamado estadio 0 o melanoma \textit{in situ} donde el tratamiento arroja una tasa de éxito de casi el 100\% ya que sólo se necesita una 
cirugía menor para extirpar la porción de piel afectada. Esto hace que la detección precoz de esta enfermedad sea una premisa para mejorar el 
tratamiento oncológico.

Las enfermedades de la piel entrenadas para el modelo son las siguientes: Melanoma (MEL), Nevus Melanocítico (NV), Carcinoma de Células Basales (BCC), 
Queratosis Actínica, Enfermedad de Bowen (carcinoma intraepitelial) (AKIEC), Queratosis Benigna (BKL), Dermatofibroma (DF), Lesión Vascular (VASC).

\begin{enumerate} 
    \item[.] Melanoma (MEL): Es una forma agresiva de cáncer de piel que puede afectar a personas de cualquier edad. Se detecta mediante la observación de cambios 
    en las manchas de la piel, como nuevos crecimientos, cambios en el tamaño, forma o color, o sangrado. 
    \item[.] Nevus Melanocítico (NV): Son manchas de piel de color oscuro que pueden ser benignas o precursoras de cáncer de piel. Se detectan mediante la 
    observación de cambios en la piel, como nuevas manchas o cambios en el tamaño, forma o color de las manchas existentes.. 
    \item[.] Carcinoma de Células Basales (BCC): Es un tipo de cáncer de piel que generalmente se encuentra en áreas expuestas a la luz solar. Se detecta mediante 
    la observación de cambios en la piel, como nuevos crecimientos, cambios en el tamaño, forma o color, o la aparición de cicatrices de escamas.
    \item[.] Queratosis Actínica: Es una enfermedad de la piel que se caracteriza por la aparición de manchas de color rojizo a anaranjado en la piel. Se detecta 
    mediante la observación de cambios en la piel, como la aparición de nuevas manchas o cambios en el tamaño, forma o color de las manchas existentes. 
    y características que indiquen la presencia de queratosis actínica.
    \item[.] Enfermedad de Bowen (carcinoma intraepitelial) (AKIEC): Es un tipo de cáncer de piel que afecta a las capas superficiales de la piel. Se detecta mediante 
    la observación de cambios en la piel, como la aparición de nuevas manchas, cambios en el tamaño, forma o color de las manchas existentes, o la aparición 
    de cicatrices de escamas.
    \item[.] Queratosis Benigna (BKL): Es una enfermedad de la piel que se caracteriza por la aparición de manchas de color blanco en la piel. Se detecta mediante 
    la observación de cambios en la piel, como la aparición de nuevas manchas o cambios en el tamaño, forma o color de las manchas existentes.
    \item[.] Dermatofibroma (DF): Es una enfermedad de la piel que se caracteriza por la aparición de manchas de color blanco en la piel. Se detecta mediante la 
    observación de cambios en la piel, como la aparición de nuevas manchas o cambios en el tamaño, forma o color de las manchas existentes link.springer.com. 
    \item[.] Lesión Vascular (VASC): Son creencias de la piel causadas por enfermedades cutanosas. Se detecta mediante la observación de cambios en la piel, como 
    la aparición de nuevas manchas o cambios en el tamaño, forma o color de las manchas existentes.
\end{enumerate}

Para la deteccion mediante imagenes de los mencionados canceres de piel se utilizan Redes Neuronales Convolucionales para procesamiento de imágenes 
específicamente del tipo EfficientNetB5, así como métodos de aprendizaje supervisado, tales como, Support Vector Machine (SVM) y Learning Rate Adjustment
(LRA). Los datos utilizados corresponden al concurso HAM10000 (Human Against Machine, con 10000 imágenes de entrenamiento), que contiene imágenes 
dermatoscópicas de múltiples fuentes de lesiones cutáneas cancerosas y no cancerosas. Estas imágenes fueron previamente segmentadas, parametrizadas y 
clasificadas.Este aprendizaje automático puede ofrecer a la detección de diferentes tipos de cáncer en un primer momento. 
