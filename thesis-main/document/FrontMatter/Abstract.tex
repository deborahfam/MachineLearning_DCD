\begin{resumen}
	El Machine Learning (ML) o aprendizaje automático (AM) es un subcampo de la Inteligencia Artificial (IA),
que se define como la capacidad de una máquina para aprender, mejorando su rendimiento. Los avances en
las tecnologías de IA y ML prometen una atención oncológica personalizada y equitativa y mejores resultados
sanitarios en general. El potencial de estas herramientas para generar conocimiento a partir de cantidades masivas
de datos es enorme, pudiendo así ayudar a tomar decisiones, que incluirán intervenciones, y tratamientos
de precisión contra el cáncer. Este estudio presenta una implementación de uno de los muchos artículos [1] que
existen sobre este tema, en el que se aplican técnicas de Machine Learning con redes neuronales convolucionales,
el modelo EfficientNetB5 y el uso de Learning Rate Adjustment. Se implementan los algoritmos mencionados y
se aplica el modelo de redes neuronales convolucionales a un Dataset de imágenes para medir la efectividad del
modelo y entrenar una IA para la detección precoz de diferentes tipos de cáncer obteniendo resultados propios.
\end{resumen}

\begin{abstract}
	Machine learning (ML) is a subfield of Artificial Intelligence (AI), which is defined as the ability of a machine
to learn, improving its performance. Advances in AI and ML technologies promise personalized and equitable
cancer care and better overall health outcomes. The potential of these tools to generate knowledge from massive
amounts of data is enormous, thus being able to help make decisions, which will include interventions, and
precision cancer treatments. This study presents a implementation of one of the many articles that exist on
this subject, in which Machine Learning techniques are applied with convolutional neural networks, 
the EfficientNetB5 model and the use of Learning Rate Adjustment. The aforementioned algorithms are implemented
and the convolutional neural network model is applied to an image Dataset to measure the effectiveness of the
model and train an AI for the early detection of different types of cancer with fresh results.
\end{abstract}